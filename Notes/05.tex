\chapter{Integration}

\begin{center}
  \begin{tikzpicture}
    \draw[->] (-1, 0) -- (4.2, 0) node[right] {$x$};
    \draw[->] (0, -1) -- (0, 4.2) node[above] {$f(x)$};
    \draw[scale=1, domain=0:4, smooth, variable=\x] plot ({\x}, {\x^2/4});
    % \draw[dotted] (1, 0) -- (1, 0.25);
    \draw[dotted] (1, 0.25) -- (2, 0.25);
    % \draw[dotted] (2, 0) -- (2, 1);
    \draw[dotted] (2, 1) -- (3, 1);
    % \draw[dotted] (3, 0) -- (3, 2.25);
    \draw[dotted] (3, 2.25) -- (4, 2.25);
    % \draw[dotted] (4, 0) -- (4, 4);

    \draw[-] (1, 0.1) -- (1, -0.1) node[below] {$\frac{1}{4}$};
    \draw[-] (2, 0.1) -- (2, -0.1) node[below] {$\frac{1}{2}$};
    \draw[-] (3, 0.1) -- (3, -0.1) node[below] {$\frac{3}{4}$};
    \draw[-] (4, 0.1) -- (4, -0.1) node[below] {$1$};


    \draw[dotted] (0, 0) rectangle (1, 0.25);
    \draw[dotted] (1, 0) rectangle (2, 1);
    \draw[dotted] (2, 0) rectangle (3, 2.25);
    \draw[dotted] (3, 0) rectangle (4, 4);
  \end{tikzpicture}

  $f(x) = x^2$ defined on $[0, 1]$
\end{center}

\begin{enumerate}[label=\arabic*.]
  \item Given an interval $[a, b]$ and a function $f$ defined by $f: [a, b] \to \RR$
  , assume that $f$ is bounded (there is an $M$ such that $|f(x)| \leq M$ for all $x \in [a, b]$)
\end{enumerate}

\begin{definition*}
  A partition of $[a, b]$ is a finite collection of distinct numbers in $[a, b]$. 
  which contains the end point $a$ and $b$.

  We can order these 
  $$a = x_0 < x_1 < x_2 < \dotsc < x_N = b$$
  Are finement $\tilde{P}$ of partition $P$, is a partition which contains $P$ 
\end{definition*}

\begin{definition*}[Lower Riemann Sum]
 $$L(f, P) = \sum_{j=1}^N (x_j - x_{j-1})\cdot m_j$$
 $$m_j = \inf_{x_{j-1}\leq t\leq x_j}f(t)$$
  Lower Riemann sum, given $f$, and $P = \{x_0 <x_1 <\dotsc< x_N\}$, 
  $\inf$ is the greatest lower bound for $f$ on $[x_{j-1}, x_j]$,
  $m_j$ is the greatest lower bound for $f$ on the partition$[x_{j-1}, x_j]$
\end{definition*}

\begin{definition*}[Uower Riemann Sum]
  $$U(f, P) =\sum_{j=1}^N (x_j - x_{j-1})\cdot M_j$$
  $$M_j = \sup_{x_{j-1}\leq t\leq x_j}f(t)$$
  where $\sup$ is the least upper bound,
  $M_j$ is the least upper bound
\end{definition*}

the example $[a, b] = [0, 1]$
$$P=\left\{0, \frac{1}{N}, \frac{2}{N}, \dotsc, \frac{N}{N} = 1\right\}$$
where $x_j = \frac{j}{N}, 0\leq j \leq N$
\begin{align*}
  L(f, P) &= \sum_{j=1}^{N}\frac{1}{N}\left(\frac{j-1}{N}\right)^2 \\
  &= \frac{1}{N^3}\sum_{j=1}^{N} (j-1)^2 \\
  &= \frac{1}{N^3}(0^2 + 1^2 + 2^2 + \dotsc + (N-1)^2) \\
% \end{align*}
% \begin{align*}
  U(f, P) &= \sum_{j=1}^{N}\frac{1}{N}\left(\frac{j}{N}\right)^2 \\
  &= \frac{1}{N^3}\sum_{j=1}^{N} (j)^2 \\
  &= \frac{1}{N^3}(1^2 + 2^2 + \dotsc + N^2) \\
\end{align*}

What happens with $L(f, P)$ if we refine the partition?

What happens with $U(f, P)$ if we refine the partition?

\underline{\textbf{Proposition}}:if $P, \tilde{P}$ are partitions $P \subset \tilde{P}$ then 
$L(f, P) \leq L(f, \tilde{P})$ and $U(f, P) \geq U(f, \tilde{P})$

\begin{center}
  \begin{tikzpicture}
    \draw[->] (-1, 0) -- (4.2, 0) node[right] {$x$};
    \draw[->] (0, -1) -- (0, 4.2) node[above] {$f(x)$};
    \draw[scale=1, domain=0:4, smooth, variable=\x] plot ({\x}, {\x^2/4});
    \draw[dotted] (1, 0) -- (1, 0.25);
    \draw[dotted] (1, 0.25) -- (2, 0.25);
    \draw[dotted] (2, 0) -- (2, 1);
    \draw[dotted] (2.5, 0) -- (2.5, 1.5625);
    \draw[dotted] (2.5, 1.5625) -- (3, 1.5625);
    \draw[dotted] (2, 1) -- (3, 1);
    \draw[dotted] (3, 0) -- (3, 2.25);
    \draw[dotted] (3, 2.25) -- (4, 2.25);
    \draw[dotted] (4, 0) -- (4, 4);

    \draw[-] (1, 0.1) -- (1, -0.1) node[below] {$\frac{1}{4}$};
    \draw[-] (2, 0.1) -- (2, -0.1) node[below] {$\frac{1}{2}$};
    \draw[-] (2.5, 0.1) -- (2.5, -0.1) node[below] {$v$};
    \draw[-] (3, 0.1) -- (3, -0.1) node[below] {$\frac{3}{4}$};
    \draw[-] (4, 0.1) -- (4, -0.1) node[below] {$1$};


    % \draw[dotted] (0, 0) rectangle (1, 0.25);
    % \draw[dotted] (1, 0) rectangle (2, 1);
    % \draw[dotted] (2, 0) rectangle (3, 2.25);
    % \draw[dotted] (3, 0) rectangle (4, 4);
  \end{tikzpicture}

  $f(x) = x^2$ defined on $[0, 1]$
\end{center}

If we refine the partition by adding a point $\tilde{P} = P \cup \{v\}$

The key for the proof $x_{j-1}, x_j \in P$ take a new partition point $v$ between $x_{j-1}$ and $x_j$
\begin{align*}
  (x_j - x_{j-1}) \inf_{x_{j-1} \leq t \leq x_j} f(t) &= (x_j - v) \inf_{x_{j-1} \leq t \leq x_j} f(t) + (v - x_{j-1}) \inf_{x_{j-1} \leq t \leq x_j} f(t) \\
  &\leq (x_j - v) \inf_{v \leq t \leq x_j} f(t) + (v - x_{j-1}) \inf_{x_{j-1} \leq t \leq v} f(t) \\
\end{align*}

\begin{theorem*}
  Let $f$ be a bounded function on $[a, b]$. Then
  $$\sup_{P} L(f, P) \leq \inf_{P} U(f, P)$$
  where $\sup_{P}$ is supremum over all partition and $\inf_{P}$ is infimum over all partitions.
\end{theorem*}

\begin{definition*}
  Given the theorem, we say that $f$ is integrable (or Riemann integrable) if
  $$\sup_{P} L(f, P) = \inf_{P} U(f, P)$$
\end{definition*}

\begin{proof}
  \text{}
  \begin{enumerate}[(i)]
    \item If $P_1, P_2$ are two partitions then 
    $$L(f, P_1) \leq U(f, P_2)$$
    Key: Take a refinement $P$ of both $P_1, P_2$ where $P \supset P_1 \cup P_2$ then 
    $$L(f, P_1) \leq L(f, P) \leq U(f, P) \leq U(f, P_2)$$
    \item Conjecture: $U(f, P_2)$ is an upper bound for all $L(f, P)$ where $P$ is any partition
    so the least upper bound for the $L(f, P)$ cannot exceed $U(f, P)$ that menas 
    $$\sup_P L(f, P) \leq U(f, P_2)$$
    for any fixed partition

    For all partition $P_2$, the number $\displaystyle\sup_P L(f, P)$ is a lower bound for $U(f, P_2)$
    The greatest lower bound for the $U(f, P_2)$ cannot be smaller than $\displaystyle\sup_P L(f, P)$
    $$\implies \sup_P L(f, P) \leq \inf_P U(f, P)$$
  \end{enumerate}
\end{proof}

\begin{example*}
\[
  f(x) = \begin{cases}
    1 & \text{if } x \text{ is irrational on } [0, 1] \\
    0 & \text{if } x \text{ is rational on } [0, 1]
    \end{cases}
\]
Take any partition $P =\{0 = x_0 < x_1 < \cdots < x_n = 1\}$ on $[x_{j-1}, x_j]$
$$\inf_{[x_{j-1}, x_j]}f = 0\text{ and }\sup_{[x_{j-1}, x_j]}f = 1$$
$$L(f, P) = \sum_{j=1}^{N} (x_j - x_{j-1}) \cdot 0 = 0$$
$$U(f, P) = \sum_{j=1}^{N} (x_j - x_{j-1}) \cdot 1 = 1$$
So this function is not integrable
\end{example*} 

\begin{example*}
\[
  f(x) = \begin{cases}
    1 & \text{if } 0 < x < \frac{1}{2}\\
    0 & \text{if } \frac{1}{2} \leq x \leq 1
    \end{cases}
\] 

$$P = \left\{0, \frac{1}{2}, 1\right\}, \inf_{[0, \frac{1}{2}]} = 0, \sup_{[0, \frac{1}{2}]} f = 1$$
$$L(f, P) = 0\cdot\frac{1}{2} + \frac{1}{2} = \frac{1}{2}$$
$$U(f, P) = 1\cdot\frac{1}{2} + \frac{1}{2} = 1$$
We pick new points
$$x_0 = 0, x_1 = \frac{1}{2} - \frac{1}{N}, x_2 = \frac{1}{2}, x_3 = 1$$
$$L(f, P) = \left(\frac{1}{2} - \frac{1}{N}\right)\cdot 0 + \frac{1}{N}\cdot 0 + \frac{1}{2}\cdot 1 = \frac{1}{2}$$
$$U(f, P) = \left(\frac{1}{2} - \frac{1}{N}\right)\cdot 0 + \frac{1}{N}\cdot 1 + \frac{1}{2}\cdot 1 = \frac{1}{2} + \frac{1}{N}$$
$$\sup L(f, P) \ge \frac{1}{2}\text{ and } \inf U(f, P) \le \frac{1}{2} + \frac{1}{N}$$
$$\implies \sup L(f, P) = \inf U(f, P) = \frac{1}{2}$$
\end{example*}

\begin{example*}
  Define on $[0, 1]$

\[
  f(x) = \begin{cases}
    0 & \text{if } x \text{ is irrational}\\
    \frac{1}{q} & \text{if } x \text{ is rational}, x = \frac{p}{q}\text{ in lowest term}
    \end{cases}
\]

Question: Is $f$ integrable on $[0, 1]$?

Answer: Yes, and the integral is $0$
\end{example*}

\section{Simple criteria for Riemann integrability}

% Criteria
\begin{theorem*}
  Let $f$ be a bounded function on $[a, b]$. Then $f$ is integrable if and only if for every $\eps > 0$
  there is a partition $P$, such that $$U(f, P) - L(f, P) < \eps$$
\end{theorem*}

Definition of Riemann integrable by if
$$\sup_{P_1} L(f_1, P_1) = \inf_{P_2} U(f_1, P_2)$$
then $f$ is Riemann integrable, and the (common) value is $$\int_{a}^{b}f $$

If $\tilde{P}$ is a refinement of $P$ then $L(f, \tilde{P}) \geq L(f, P)$ and $U(f, \tilde{P}) \leq U(f, P)$

\begin{proof}
  \text{}
  \begin{enumerate}[1.]
    \item Assume that $f$ is Riemann integrable.
    Let $\eps > 0$ there is a partition $P_1$ such that 
    $$\int_a^b f - \frac{\eps}{10} < L(f, P_1)$$
    There is a partition $P_2$ such that 
    $$U(f, P_2) < \int_a^b f+ \frac{\eps}{10}$$
    Then we get 
    $$\int_a^b f - \frac{\eps}{10} < L(f, P_1) \leq L(f, P_1\cup P_2) \leq U(f, P_1\cup P_2) \leq U(f, P_2) < \int_a^b f+ \frac{\eps}{10}$$
    $$\implies U(f, P_1 \cup P_2) - L(f, P_1 \cup P_2) < \frac{\eps}{5}$$
    \item Assume that for every $\eps > 0$, there is a partition $P$ such that
    $$U(f, P) - L(f, P) < \eps$$
    We want to show that $\displaystyle\inf_P U(f, P) - \displaystyle\sup_P L(f, P)$ differs by no more than $\eps$
    If I succeed, then we get (since $\eps > 0$ is arbitrary) that 
    $\displaystyle\inf_P U(f, P) = \displaystyle\sup_P L(f, P)$
    (i.e. by definition $f$ is Riemann integrable)

    We here shown $\inf U(f, P) \geq \sup L(f, P)$
    $$U(f, P_\eps) \ge \inf_{\text{all }P}U(f, P) \ge \sup _{\text{all }P}L(f, P) \ge L(f, P_\eps)$$
  \end{enumerate}
\end{proof}
\section{Continuous functions}

\begin{theorem*}
  A continuous function $f$ on $[a, b]$ is Riemann integrable
\end{theorem*}

Recall the defnition of continuity.
A function is continuous at $x_0 \in [a, b]$
$$\forall \eps > 0, \exists \delta > 0, |x-x_0| < \delta \implies |f(x) - f(x_0)| < \eps\ (\text{for }x \in [a, b])$$
Theorem: A continuous function on $[a, b]$ is uniformly continuous.

\begin{proof}
  $f$ is uniformly continuous, we \underline{want to check} for arbitrary $\eps > 0$
  that there is a partition $P$ such that $U(f, P) - L(f, P) < \eps$

  \underline{Know:} There is a $\delta > 0$ such that $|f(x_1) - f(x_2)| < \frac{\eps}{b-a}$ provided That $|x_1 - x_2| < \delta$

  $$P = \left\{x_j = a + j\frac{\eps}{b-a},\right\}, x_j-x_{j-1} = \frac{\eps}{b-a} = \delta$$

  \begin{align*}
    U(f, P) - L(f, P) &= \sum_{j=1}^{N} M_j \cdot (x_j - x_{j-1}) - \sum_{j=1}^{N} m_j \cdot (x_j - x_{j-1})\\
    &= \sum_{j=1}^{N} \underbrace{(M_j - m_j)}_{\leq \frac{\eps}{b-a}\cdot \frac{1}{100}} \cdot (x_j - x_{j-1})\\
    &\leq \sum_{j=1}^{N} \frac{\eps}{b-a}\cdot \frac{1}{100} \cdot (x_j - x_{j-1})\\
    &\leq \frac{\eps}{b-a}\cdot \frac{1}{100} \cdot (b-a)\\
    &= \frac{\eps}{100}
  \end{align*}
\end{proof}
