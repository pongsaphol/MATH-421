\chapter{Differenitiation}

Office hours on Monday

\begin{enumerate}
  \item Office hour 6.pm to 7.pm on Monday
  \item can meet before 8:50 am Monday in my office Van Vleck 613 (send an email on sunday)
\end{enumerate}

Consider a function defined on on interval $I$, with real values.
$f: I \to \RR$

\begin{definition*}
  $f$ is differentiable at the point $a \in I$ if the limit $\lim\limits_{x \to a} \frac{f(x) - f(a)}{x - a}$  exists,
  then we call this limit the deriviative $f'(a)$
\end{definition*}

\begin{center}
  \begin{tikzpicture}
    \draw[->] (-1, 0) -- (4.2, 0) node[right] {$x$};
    \draw[->] (0, -1) -- (0, 4.2) node[above] {$f(x)$};
    \draw[scale=1, domain=0.3:3.5, smooth, variable=\x] plot ({\x}, {\x^2/4});

    \draw[dotted] (2, -0.5) -- (2, 4) node[above] {$$};
    \draw[dotted] (2.5, -0.5) -- (2.5, 4) node[above] {$$};
    \draw[-] (2, 1) -- (2.5, 1) node[above] {$$};
    \draw[-] (2.5, 1) -- (2.5, 1.5625) node[above] {$$};
    \node[] at (2.25,0.8) {$x-a$};
    \node[] at (4,1.25) {$f(x) - f(a)$};
    \node[] at (2,-0.2) {$a$};
    \node[] at (2.5,-0.2) {$x$};
  \end{tikzpicture}

  $y = f(x)$, $\frac{f(x) - f(a)}{x-a} = $ slope of $f$
\end{center}

Computation of some derivatives

\begin{example*}
  \text{ }
  \begin{enumerate}[(i)]
    \item $f(x) = c$ ($c$ is some fixed point)
      we get $f'(a) = 0$ for all $a$, 

      $f(x) = f(a) = 0$ for all $x$, $\frac{f(x) - f(a)}{x-a} = 0 \implies f$ is differentiable and $f'(a) = 0$ for all $a$

      $\lim\limits_{x \to a} \frac{f(x) - f(a)}{x-a} = f'(x)$ is equivalent with saying $\lim\limits_{h \to 0} \frac{f(a+h) - f(a)}{h} = f'(a)$
    \item $f(x) = x$, then $$\frac{f(a+h) - f(a)}{h} = \frac{a + h - a}{h} = 1$$ (written $f'(x) = 1$)

    \item $f(x) = x^2$, then fix $a$, $$\frac{f(a + h) - f(a)}{h} = \frac{(a+h)^2 - a^2}{h} = \frac{a^2 + 2ah + h^2 - a^2}{h} = 2a + h$$
    $$\lim\limits_{h \to 0} \frac{f(a+h) - f(a)}{h} = \lim\limits_{h \to 0}2a + h = 2a$$
    \item $f(x) = |x|$, We should examine the differentiability of $f$ at \underline{$a = 0$}
    $$\frac{f(0 + h) - \overbrace{f(0)}^{=0}}{h} = \frac{|h|}{h} =\begin{cases}
      1       & \quad \text{if } h > 0\\
      -1  & \quad \text{if } h < 0
    \end{cases}$$
    The limit does not exist, and thus $f$ is not differentiable at $0$.
    \item $f(x) = \sqrt{|x|}$, $f$ is not differentiable at $0$ because $f(0) = 0$ and $\frac{f(0 + h) - f(0)}{h} = \frac{\sqrt{|h|}}{h}$, this limit also does not exist 

    Examine differentiability and derivative of $f(x) = \sqrt{|x|}$ at $x = a, a > 0$
    \begin{align*}
    \frac{f(a + h) - f(a)}{h} &= \frac{\sqrt{|a + h|} - \sqrt{|a|}}{h} \\
    &= \frac{\sqrt{a + h} - \sqrt{a}}{h} \\
    &= \frac{a+h - a}{\sqrt{a + h} + \sqrt{a}}\cdot \frac{1}{h} \\
    &= \frac{1}{\sqrt{a + h} + \sqrt{a}} \to \frac{1}{2\sqrt{a}}\\
    \end{align*}
    \item $f(x) = x^n$ 
    $$\frac{f(a+h) - f(a)}{h} = \frac{(a + h) ^ n - a^n}{h} = n\cdot a^{n-1}$$
  \end{enumerate}
\end{example*}

\section{Basic fact about differentiation}

Continuity is necessary (but not sufficient) for differentiation 

\begin{theorem*}
  If $f: I \to \RR$ is differentiable at $a$ the $f$ is continuous at $a$.
\end{theorem*}

\textbf{Reminder} If $\lim\limits_{x \to a} F(x) = l$ and $\lim\limits_{x \to a} G(x) = m$, then $\lim\limits_{x \to a} F(x)G(x)= lm$

If $\lim\limits_{x \to a} F(x) = l$ and $\lim\limits_{x \to a} G(x) = m$, then $\lim\limits_{x \to a} \frac{F(x)}{G(x)}= \frac{l}{m}$ or not? Yes if $m \neq 0$

\begin{proof}
  We know that $\lim\limits_{x \to 0} \frac{f(a + h) - f(a)}{h} = f'(a)$
  $$f(a + h) - f(a) = \frac{f(a + h) - f(a)}{h} \cdot h$$
  $$\implies \lim\limits_{h \to 0} f(a + h) - f(a) = f'(a) \cdot 0 = 0$$
  $$\lim\limits_{h \to 0} f(a + h) = f(a)$$ 
  this is continuity of $f$ at $a$
  
  Another argument : for sufficiently small $h$, $|f(a + h) - f(a)| \leq C|h|$
\end{proof}

\section{Sum Rule}

\begin{theorem*}
  Let $f: I \to \RR$ and $g: I \to \RR$, $a\in I$ assume that $f$ and $g$ are differentiable at $a$. 
  Then $f + g$, $(f + g)(x) = f(x) + g(x)|_{x = a}$ is differentiable and its derivative $f'(a) + g'(a)$
  (The derivative of the sum is the sum of the derivatiives)
\end{theorem*}

\begin{proof}
  \begin{align*}
    \frac{(f+g)(a + h) - (f+g)(a)}{h} &= \frac{f(a + h) + g(a + h) - (f(a) + g(a))}{h} \\
    &= \frac{f(a + h) - f(a)}{h} + \frac{g(a + h) - g(a)}{h} \\
  \end{align*}
  As $h \to 0$ this has limit $f'(a) + g'(a)$
\end{proof}

\section{Product Rule}

\begin{theorem*}
  Let $f: I \to \RR$ and $g: I \to \RR$, $a\in I$ assume that $f$ and $g$ are differentiable at $a$. 
  the $f\cdot g$ is differentiable at $a$
  $$(f\cdot g)'(a) = f'(a)g(a) + f(a)g'(a)$$
\end{theorem*}
\begin{proof}
  \begin{align*}
    \frac{f(a + h)g(a + h) - f(a)g(a)}{h} &= \frac{(f(a+h) - f(a))g(a+h) + f(a)g(a+h) - f(a)g(a)}{h} \\
    &= \underbrace{\frac{f(a+h) - f(a)}{h}}_{f'(a)}\cdot \underbrace{g(a+h)}_{\to g(a)} + \underbrace{\frac{g(a + h) - g(a)}{h}}_{\to g'(a)} \cdot \underbrace{f(a)}_{f(a)} \\
  \end{align*}
  By theorem about products and of limits, and the continuity of $g$ at $a$, we get

  $$\lim\limits_{h \to 0} \frac{f(a+h)g(a+h) - f(a)g(a)}{h} = f'(a)g(a) + g'(a)f(a)$$
\end{proof}


\begin{theorem*}
  Let $f: I \to \RR$ and $g: I \to \RR$, $a\in I$ is differentiable at $a$, and if $g(a) \neq 0$
  then $\frac{1}{g}$ is differentiable at $a$ and $$\left(\frac{1}{g}\right)'(a) = -\frac{g'(a)}{(g(a))^2}$$
  % $$\left(\frac{f}{g}\right)'(a) = \frac{f'g - fg'}{g^2} |_a$$
\end{theorem*}
\begin{proof}
  \begin{align*}
    \frac{\frac{1}{g(a+h)} - \frac{1}{g(a)}}{h} &= \frac{g(a) - g(a + h)}{g(a+h)g(a)}\cdot\frac{1}{h} \\
    &= \frac{1}{g(a+h)g(a)}\cdot (-1)\frac{g(a+h) - g(a)}{h} \\
    &\to \frac{1}{(g(a))^2}\cdot (-1) g'(a) 
  \end{align*}
\end{proof}


\section{Quotient Rule}
\begin{theorem*}
  Let $f: I \to \RR$ and $g: I \to \RR$, $a\in I$ assume $f$ and $g$ are differentiable at $a$, and if $g(a) \neq 0$
  then $\frac{f}{g}$ is differentiable at $a$ and $$\left(\frac{f}{g}\right)'(a) = \frac{f'(a)g(a) - f(a)g'(a)}{g(a)^2}$$
\end{theorem*}
\begin{proof}
  Combine the theorems about products and reciprocals of differentiable function

  \begin{align*}
  \left(\frac{f}{g}\right)'(a) &= \left(f \cdot \frac{1}{g}\right)'(a) \\
  &= f'(a) \cdot \frac{1}{g(a)} + f(a) \cdot \left(\frac{1}{g}\right)'(a) \\
  &= \frac{f'(a)}{g(a)} + f(a)\left(-\frac{g'(a)}{g(a)^2}\right) \\
  &= \frac{f'(a)g(a) - f(a)g'(a)}{g(a)^2}
  \end{align*}
\end{proof}

\begin{example*}
  \begin{align*}
    \left(\frac{\sin x}{\cos x}\right)' &= \frac{\sin'(x) \cos(x) - \cos' x \sin x}{(\cos x)^2} \\
    &= \frac{(\cos x)^2 - (-\sin ^2 x)}{(\cos x)^2} \\
    &= \frac{1}{(\cos x)^2} \\
  \end{align*}
\end{example*}