\chapter{Properties of Real Number}

\section{Addition Properties}

\begin{enumerate}[1.]
  \item (Associativity) Given $a, b, c \in \RR$ then $(a+b)+c = a+(b+c)$
  \item (Additive Identity) There exists the number zero denoted by $0$ which $a + 0 = 0 +a = a, a \in \RR$
  \item (Additive Inverse) Given any $a \in \RR$ there exists a number $(-a) \in \RR$ such that $a + (-a) = (-a) + a = 0$
  \item (Communicativity) Given any $a, b \in \RR$ then $a + b = b + a$
\end{enumerate}

$(\RR, +)$ forms a communicative group

\section{Multiplication Properties}
\begin{enumerate}[1.]
  \setcounter{enumi}{4}
  \item (Associativity) Given $a, b, c \in \RR$ then $(a\cdot b)\cdot c = a\cdot (b\cdot c)$
  \item (Additive Identity) There exists the number one denoted by $1$ which $a \cdot 1 = 1 \cdot a = a, a \in \RR$
  \item (Additive Inverse) Given any $a\neq 0, a \in \RR$ there exists a number $a^{-1} \in \RR$ such that $a \cdot a^{-1} = a^{-1} \cdot a= 1$
  \item (Communicativity) Given any $a, b \in \RR$ then $a \cdot b = b \cdot a$
\end{enumerate}
$(\RR \setminus \{0\}, \cdot)$ forms a communicative group
\begin{enumerate}[1.]
  \setcounter{enumi}{8}
  \item (Distributivity) Given $a, b, c \in \RR$ then $a\cdot(b+c) = a\cdot b + a\cdot c$
\end{enumerate}

A special propertiy of $\RR$ is that it has an \textbf{ordinary}, i.e., we use the symbols $\ge, >, <, \le$

\begin{enumerate}
  \item if $a > 0$, we say $a$ is positive
  \item if $a \ge 0$, we say $a$ is non-negative
  \item if $a < 0$, we say $a$ is negative
  \item if $a \le 0$, we say $a$ is non-positive
\end{enumerate}

\begin{definition}
Given any $a\in\RR$, we define its absolute value to be
\[ |a| =
  \begin{cases}
    a       & \quad \text{if } a \geq 0 \\
    a  & \quad \text{if } a < 0
  \end{cases}
\]
\end{definition}

\begin{theorem}[Triangular Inequality]
Given $a, b\in\RR$, there holds
$$|a+b|\leq |a| + |b|$$
\end{theorem}

\begin{proof}
  There are 4 cases to consider
  \begin{enumerate}[(i)]
    \item $a \ge 0, b \ge 0$
    \item $a \ge 0, b < 0$
    \item $a < 0, b \ge 0$
    \item $a < 0, b < 0$ 
  \end{enumerate}
  \begin{enumerate}[step 1.]
    \item since $a+b = b+a$, by exchanging the rule of $a$ and $b$ we see that (ii) $\iff$ (iii)

    the case $a = 0$ in (i) is easy, since $|0 + b| = |b| = |b| + 0$ and so the inequality is satisfied

    Moreover, when $a < 0, b < 0$, as in (iv), we have $(-a) > 0$ and $(-b) > 0$ and 
    \[|a+b| = |-(a+b)| = |(-a)+(-b)|\] 
    while $|-a| = |a|$ and $|-b| = |b|$, satisfies to prove (i)

    So, it satifies to prove (i) and (ii)
    \item We prove (i), since $a + b \ge 0$ when $a, b \ge 0$, so $|a + b| = a + b = |a| + |b|$, so, the inequality is satisfied
    \item We prove the case which $a \ge 0, b < 0$, the case can be written in 2 subcases
    \begin{enumerate}[(A)]
      \item $a + b \ge 0$
      \item $a + b < 0$
    \end{enumerate}
    Suppose (A) is true. then $|a + b| = a + b = |a| + b \le |a| + |b|$

    Case (b) $|a +b| = -(a + b) = (-a) + (-b) \le |a| + |b|$
  \end{enumerate}
\end{proof}
\section{Notation}
\begin{itemize}
  \item a set is a collection of number, say $\{0, \pi, -101\}$
  We call the set $A$, for instance, and write $A = \{0, \pi, -101\}$

  Given some set $B$, we write $x\in B$ if the number $x$ belongs to the set $B$, 
  $y \notin B$ if the number $y$ does not belong to the set $B$. 
  For instance, $0 \in A$ and $2 \notin A$
  \item The collection of natural numbers is denoted by
  \[\NN = \{1, 2, 3, \dotsc\}\]
  \item There is also the collection of integers, denoted by
  \[\ZZ = \{1, 2, 3, \dotsc\} \cup \{0\} \cup \{-1, -2, -3, \dotsc\}\]
  \item The collection of rational numbers, denoted by
  \[\QQ = \left\{\frac{m}{n} : m \in \ZZ; n \in \NN\right\}\]
  \item The symbol $\setminus$ means ``minus'', e.g., if $A = \{0, 1, 2, 3, \pi\}$ and $B = \{1, \pi\}$ then $A \setminus B = \{0, 2, 3\}$
  \item We denote the irrational numbers by $\RR \setminus \QQ$
  \item within $\NN$, we can distinguish between the even numbers and the odd numbers, 
  \begin{enumerate}
    \item[even] $\{p\in\NN : p = 2q,\ \exists q \in \NN\}$
    \item[odd] $\{p\in\NN : p = 2q-1,\ \exists q \in \NN\}$
  \end{enumerate}
\end{itemize}