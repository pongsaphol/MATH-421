\chapter{Method of Proof}
\section{Direct proof}
some statements can be shown to be true through a direct arguement e.g. our proof of Theorem 1

\section{Proving the contrapositive}
Let $A$ and $B$ be statements. Suppose we are trying to show that $A$ implies $B$, which is written $A \implies B$,
This is equivalent to show that $\text{not }B \implies \text{not }A$, which is written $\lnot B \implies \lnot A$.

\section{Proof by contradiction}
To prove $A \implies B$. Start by assuming that $B$ is false and show that this leads to a contradiction in $A$.

\section{Proof by induction}
the aim is to proof that a statement is true for all rational number
\begin{enumerate}[(i)]
    \item Show the statement is true for $n=1$
    \item Assume the statement is true for general $n\in \NN$
    \item Using assumption (ii), prove the statement is true for $n+1$
    \item Conclude your proof with a sentence like "by mathematical information, the result holds for all $n\in \NN$"
\end{enumerate}
\begin{example}
    Show that $\sqrt{2} \in \RR\setminus\QQ$
\end{example}
\begin{example}
    for $n \in \NN$ and $k \in \ZZ$, $\binom{n}{k}$ is denoted by
    \[\binom{n}{k} = \begin{cases}
        \frac{n!}{k!(n-k)!} & \text{if } k \le n \\
        0 & \text{otherwise}
    \end{cases}\]
    There holds
    \begin{enumerate}[(a)]
        \item $\displaystyle\binom{n+1}{k} = \binom{n}{k-1} + \binom{n}{k}$
        \item $\displaystyle\binom{n}{k}$ belongs to $\NN$ for all choices of $n$ and $k$
        \item (binomial theorem) for all $x \in \RR$ we have 
        $$(1+x)^n=\sum_{j=0}^{n} \binom{n}{j}x^j$$
    \end{enumerate}
\end{example}

\begin{theorem}[binomial theorem]
    Let $x\in \RR$ and $n \in \NN$. Then, there holds the formula
    $$(1+x)^n=\sum_{j=0}^{n} \binom{n}{j}x^j$$
\end{theorem}
