\chapter{Taylor's Theorem}

\begin{example*}
  Our fundamental theorem of calculus
  \[\int_a^x f'(t)\,dt = f(x) - f(a)\]

  Meanwhile theorem for integrals
  \[f(x) = f(a) = (x-a)f'(\xi)\]
  (also we get this from the mean value theorem for derivatives)

  Do an integration by parts, in the integral (If $f$ is differentiable twice)
  \begin{align*}
    \int_a^x 1 f'(t)\, dt &= \int_a^x \underbrace{\frac{d}{dt}(t-x)}_{u'}\underbrace{f'(t)}_v\, dt \\
    &= \underbrace{(x-x)f'(x)}_{=0} - (a-x)f'(a) - \int_a^x (t-x)f''(t)\, dt \\
    &= (x-a)f'(a) + \int_a^x (x-t)f''(t)\, dt
  \end{align*}
  \[f(x) - f(a) = f'(a)(x-a) + \int_a^x(x-t)f''(t)\, dt\]
  View this formula as approximating $f(x)$ for $x$ near $a$ by $f(a) + f'(a)(x-a) + \text{ Error term}$

  Error term $\displaystyle = \int_a^x (x-t) f''(t)\, dt$
\end{example*}

\section{Taylor's theorem on polynomial approximation}

\begin{itemize}
  \item Given a function defined on an interval $I$ and $a \in I$
  \item (Assumptions), $f$ is differentiable $n + 1$-times on $I$, let says $f^{(n+1)}$ is continuous on $I$
\end{itemize}
Therefore, 
\[f(x) = \underbrace{\sum_{k = 0}^{n} \frac{f^{(k)}(a)}{k!}(x-a)^k}_{T_N(x, a)} + \underbrace{R_n(x, a)}_{\text{``error term''}}\]
\[R_n(x, a) = \frac{(x-a)^{n+1}}{n!}\int_0^1(1-t)^nf^{(n+1)}(a + t(x-a))\, dt\]
It can also be written as 
\[\frac{1}{n!}\int_a^x (x-u)^nf^{(n+1)}(u)\, du\]

\[T_n(x, a) = f(a) + f'(a)(x-a) + \frac{f''(a)}{2!}(x-a)^2 + \cdots + \frac{f^{(n)}(a)}{n!}(x-a)^n\]
Convention $0! = 1$

Factorial can be written in integral term
\[n! = \int_0^\infty t^ne^{-t}\, dt = \lim\limits_{R\to\infty} \int_0^{R}t^ne^{-t}\, dt\]
\[\Gamma(x) = \int_0^\infty t^{x-1}e^{-t}\, dt\]

If $f$ is itself a polynomial of degree $\le n$ then $R_n(x, a) = 0$

Two form of the remainder are the same, in the second form 
\begin{align*}
  u &= a + t(x-a) (u\text{ is between } a \text{ and } x \iff t \text{ between 0 and 1})\\ 
  du &= (x-a)\, dt
\end{align*}
\begin{align*}
  \int_a^x(x-u)^nf^{(n+1)}(u)\, du &= \int_0^1(x-a -t(x-a))^nf^{(n+1)}(a + t(x-a))(x-a)\, dt \\ 
  &= \int_0^1\left[(x-a)(1-t)\right]^nf^{(n+1)}(a + t(x-a))(x-a)\, dt \\
  &= (x-a)^{n+1} \int_0^1(1-t)^nf^{(n+1)}(a + t(x-a))\, dt
\end{align*}

\begin{proof}
  Checking that 
  \[R_n(x, a) = \frac{f^{n_1}(a)}{(n+1)!}(x-a)^{n+1} R_{n+1}(x, a)\]
  Proof by Induction

  $n = 0$
  \begin{align*}
    f(x) &= f(0) + \frac{1}{0!}\int_a^x(x-a)^0f'(u)\, du\\
    &= f(0) + \int_a^x f'(u) \, du\\
  \end{align*}
  True by FTC
  \[U(u) = -\frac{(x-u)^{n+1}}{n+1}\]
  \[V(u) = f^{(n+1)}(u)\]
  \begin{align*}
    R_n(x, a) &= \frac{1}{n!}\int_a^x\underbrace{(x-u)^n}_{U'(u)}\underbrace{f^{(n+1)}(u)}_{V(u)}\, du\\
    &= \frac{1}{n!}\left[\underline{U(x)V(x)}_{=0}-U(a)V(a) -\int_a^x-\frac{(x-u)^{n+1}}{n+1}f^{(n+2)}(u)\, du\right] \\
    &= \frac{(x-a)^{n+1}}{(n+1)!} + \int_a^x (x-u)^{n+1}f^{(n+2)}(u)\, du \\
    &=  \frac{(x-a)^{n+1}}{(n+1)!} + R_{n+1}(x, a) 
  \end{align*}
\end{proof}

In Spivak's book, there is a different representation of the remainder term
\[R_n(x, a) = \frac{(x-a)^{n+1}}{(n+1)!}f^{(n+1)}(\xi)\]
where $\xi$ is between $a$ and $x$

Apply the second mean value 23, ch.13

\[\int_a^x (x-u)^nf^{(n+1)}u\, du = f^{(n+1)}(\xi) \underbrace{\int_a^x (x-u)^n \, du}_{\frac{(x-a)^{n+1}}{n+1}}\]