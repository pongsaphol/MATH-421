\chapter{Taylor's Theorem}

\begin{example*}
  Our fundamental theorem of calculus
  \[\int_a^x f'(t)\,dt = f(x) - f(a)\]

  Meanwhile theorem for integrals
  \[f(x) = f(a) = (x-a)f'(\xi)\]
  (also we get this from the mean value theorem for derivatives)

  Do an integration by parts, in the integral (If $f$ is differentiable twice)
  \begin{align*}
    \int_a^x 1 f'(t)\, dt &= \int_a^x \underbrace{\frac{d}{dt}(t-x)}_{u'}\underbrace{f'(t)}_v\, dt \\
    &= \underbrace{(x-x)f'(x)}_{=0} - (a-x)f'(a) - \int_a^x (t-x)f''(t)\, dt \\
    &= (x-a)f'(a) + \int_a^x (x-t)f''(t)\, dt
  \end{align*}
  \[f(x) - f(a) = f'(a)(x-a) + \int_a^x(x-t)f''(t)\, dt\]
  View this formula as approximating $f(x)$ for $x$ near $a$ by $f(a) + f'(a)(x-a) + \text{ Error term}$

  Error term $\displaystyle = \int_a^x (x-t) f''(t)\, dt$
\end{example*}

\section{Taylor's theorem on polynomial approximation}

\begin{itemize}
  \item Given a function defined on an interval $I$ and $a \in I$
  \item (Assumptions), $f$ is differentiable $n + 1$-times on $I$, let says $f^{(n+1)}$ is continuous on $I$
\end{itemize}
Therefore, 
\[f(x) = \underbrace{\sum_{k = 0}^{n} \frac{f^{(k)}(a)}{k!}(x-a)^k}_{T_n(x, a)} + \underbrace{R_n(x, a)}_{\text{``error term''}}\]
\[R_n(x, a) = \frac{(x-a)^{n+1}}{n!}\int_0^1(1-t)^nf^{(n+1)}(a + t(x-a))\, dt\]
It can also be written as 
\[R_n(x, a) = \frac{1}{n!}\int_a^x (x-u)^nf^{(n+1)}(u)\, du\]
where $u = a + t(x-a)$

\[T_n(x, a) = f(a) + f'(a)(x-a) + \frac{f''(a)}{2!}(x-a)^2 + \cdots + \frac{f^{(n)}(a)}{n!}(x-a)^n\]
Convention $0! = 1$

Factorial can be written in integral term
\[n! = \int_0^\infty t^ne^{-t}\, dt = \lim\limits_{R\to\infty} \int_0^{R}t^ne^{-t}\, dt\]
\[\Gamma(x) = \int_0^\infty t^{x-1}e^{-t}\, dt\]

If $f$ is itself a polynomial of degree $\le n$ then $R_n(x, a) = 0$

Two form of the remainder are the same, in the second form 
\begin{align*}
  u &= a + t(x-a) (u\text{ is between } a \text{ and } x \iff t \text{ between 0 and 1})\\ 
  du &= (x-a)\, dt
\end{align*}
\begin{align*}
  \int_a^x(x-u)^nf^{(n+1)}(u)\, du &= \int_0^1(x-a -t(x-a))^nf^{(n+1)}(a + t(x-a))(x-a)\, dt \\ 
  &= \int_0^1\left[(x-a)(1-t)\right]^nf^{(n+1)}(a + t(x-a))(x-a)\, dt \\
  &= (x-a)^{n+1} \int_0^1(1-t)^nf^{(n+1)}(a + t(x-a))\, dt
\end{align*}

\begin{proof}
  Checking that 
  \[R_n(x, a) = \frac{f^{(n+1)}(a)}{(n+1)!}(x-a)^{n+1} + R_{n+1}(x, a)\]
  Proof by Induction

  $n = 0$
  \begin{align*}
    f(x) &= f(a) + \frac{1}{0!}\int_a^x(x-a)^0f'(u)\, du\\
    &= f(a) + \int_a^x f'(u) \, du\\
  \end{align*}
  True by FTC

  Let
  \[U(u) = -\frac{(x-u)^{n+1}}{n+1}\]
  \[V(u) = f^{(n+1)}(u)\]
  \begin{align*}
    R_n(x, a) &= \frac{1}{n!}\int_a^x\underbrace{(x-u)^n}_{U'(u)}\underbrace{f^{(n+1)}(u)}_{V(u)}\, du\\
    &= \frac{1}{n!}\left[\underbrace{U(x)V(x)}_{=0}-U(a)V(a) -\int_a^x-\frac{(x-u)^{n+1}}{n+1}f^{(n+2)}(u)\, du\right] \\
    &= \frac{(x-a)^{n+1}}{(n+1)!}f^{(n+1)}(u) + \int_a^x (x-u)^{n+1}f^{(n+2)}(u)\, du \\
    &=  \frac{(x-a)^{n+1}}{(n+1)!}f^{(n+1)}(u) + R_{n+1}(x, a) 
  \end{align*}
\end{proof}

In Spivak's book, there is a different representation of the remainder term
\[R_n(x, a) = \frac{(x-a)^{n+1}}{(n+1)!}f^{(n+1)}(\xi)\]
where $\xi$ is between $a$ and $x$

Apply the second mean value 23, ch.13

\[\int_a^x (x-u)^nf^{(n+1)}(u)\, du = f^{(n+1)}(\xi) \underbrace{\int_a^x (x-u)^n \, du}_{\frac{(x-a)^{n+1}}{n+1}}\]

\section{Exponential Function}

$\exp$ it is the unique differentiable function such $\exp'(x) = \exp(x)$ and $\exp(0) = 1$

For $x > 0$, $\displaystyle\log(x) = \int_1^x \frac{1}{t}\, dt$, $\log: (0, \infty) \to \RR, \log'(x) = \frac{1}{x} > 0$

$\log^{-1} = \exp, \exp(x) = e^x, \exp(1) = e$

We say $e$ is the unique number $b > 1$ for which $\displaystyle\int_1^b \frac{1}{t}\, dt = 1$

Taylor's formula for the Exponential function

\[e^x = P_n(x) + R_n(x, 0)\]
\[P_n(x) = \sum_{k=0}^n \frac{1}{k!}x^k\]
\[R_n(x, 0) = \frac{x^{n+1}}{(n+1)!}e^{\xi}\]
where $\xi$ is between $0$ and $x$

\begin{theorem*}
  The number $e$ is irrational

\end{theorem*}
\begin{proof}
  Assume that $e = \frac{p}{q}$ with integers $p, q$ where $q \ge 3$
  (we want to show a contraction to this assumption)

  \begin{align*}
    e^1 &= \sum_{k=0}^{q} \frac{1}{k!}1^k + \frac{e^\xi}{(q+1)!}1^{q+1} \text{ and $\xi$ is between $0$ and $1$ }\\
    \frac{p}{q}&= \sum_{k=0}^{q} \frac{1}{k!} + \frac{e^\xi}{(q+1)!}\\
    \frac{p}{q}q! &= q!\sum_{k=0}^{q} \frac{1}{k!} + \frac{e^\xi}{(q+1)}\\
  \end{align*}
  $\frac{pq!}{q}$ is an integer.
  $\frac{q!}{k!}$ is an integer for $k \le q$.
  Since $e^\xi < 3$ then $0 < \frac{e^\xi}{q+1} < 1$
\end{proof}
You could prove with a similar argument that $\cos 1, \sin 1$ are irrational

\[e^x = 1 + x + \frac{x^2}{2!} + \frac{x^3}{3!} + \dotsc + \frac{x^n}{n!} + R_n(x, 0)\]
\[\cos^{(k)}(0) = \begin{cases}
  \cos(0) = 1 & k = 4p\\
  -\sin(0) = 0 & k = 4p + 1\\
  -\cos(0) = -1 & k = 4p + 2\\
  \sin(0) = 0 & k = 4p + 3
\end{cases}\]
\[\cos x = 1 - \frac{x^2}{2!} + \frac{x^4}{4!} - \frac{x^6}{6!} + \dotsc \]
\[\sin^{(k)}(0) = \begin{cases}
  \sin(0) = 0 & k = 4p\\
  \cos(0) = 1 & k = 4p + 1\\
  -\sin(0) = 0 & k = 4p + 2\\
  -\cos(0) = -1 & k = 4p + 3
\end{cases}\]
\[\sin x = x - \frac{x^3}{3!} + \frac{x^5}{5!} - \frac{x^7}{7!} + \dotsc \]
Formally
\begin{align*}
  e^{ix} &= 1 + ix + \frac{(ix)^2}{2!} + \frac{(ix)^3}{3!} + \dotsc \\
  &= 1 - \frac{x^2}{2!} + \frac{x^4}{4!} - \frac{x^6}{6!} + \frac{x^8}{8!} + \dotsc + i\left( x - \frac{x^3}{3!} + \frac{x^5}{5!} + \dotsc \right)\\
  e^{ix} &= \cos x + i\sin x
\end{align*}

\begin{example*}
  $x > 0$
  \[\int_0^x \frac{\sin t}{t}\, dt\]

  Can we approximate this by a polynomial with a good estimate on the error?
\end{example*}

Strategy: Use Taylor's formula for the integral (in a smart way)

\begin{align*}
  \sin x &= x - \frac{x^3}{3!} + R_3(x, 0), R_3(x) = \frac{\sin^{(4)}(\xi)}{4!}x^4 \\
  &= x - \frac{x^3}{3!} + R_4(x, 0), R_4(x) = \frac{\cos^{(5)} (\xi)}{5!}x^5 \\ 
\end{align*}

\begin{align*}
  \int_0^x \frac{\sin t}{t}\, dt &= \int_0^x \frac{1}{t}\left(t - \frac{t^3}{6} + R_5(t, 0) \right)\, dt \\
  &= \int_0^x \left(1 - \frac{t^2}{6}\right)\,dt + \int_0^x \frac{R_5(t, 0)}{t}\, dt \\
  &= x-\frac{x^3}{16} + E(x) \\
  \frac{|R_5(t, 0)|}{|t|}&\le \frac{|t|^5}{5!}\cdot\frac{1}{|t|} = \frac{t^4}{5!} 
\end{align*}

For $x \ge 0$
\[\left|\int_0^x \frac{R_5(t, 0)}{t} \, dt \right|\le \int_0^x \frac{t^4}{5!} = \frac{x^5}{5\cdot5!} = \frac{x^5}{600}\]
For $x \le 0$

\[\left|\int_0^x \frac{R_5(t, 0)}{t} \, dt\right| \le \frac{|x|^5}{600}\]




% \[\gamma = \sum_{k=1}^n \frac{1}{k} - \ln n\]
% (Euler Mascheroni-constant) Q: Is $\gamma$ irrational?