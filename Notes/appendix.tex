\chapter{Definition and Theorem}

\section{Infremum and Supremum}

\begin{definition*}
  A set $B \subseteq \RR$ is bounded below if there exists $b \in \RR$ such that $ x \ge b$ for all $x \in B$. 
\end{definition*}

\begin{definition*}
  A set $A \subseteq \RR$ is bounded above if there exists $a \in \RR$ such that $ x \le a$ for all $x \in A$. 
\end{definition*}

\begin{definition*}
  Let $B \subseteq \RR$ be a bounded set. We say that $b \in \RR$ is the least upper bound of $B$ ($\sup B$) if
  \begin{enumerate}
    \item $b$ is an upper bound of $B$.
    \item if $b'$ is an upper bound of $B$, then $b \le b'$.
  \end{enumerate}
\end{definition*}

\begin{definition*}
  Let $A \subseteq \RR$ be a bounded set. We say that $a \in \RR$ is the greatest lower bound of $A$ ($\inf A$) if
  \begin{enumerate}
    \item $a$ is an lower bound of $A$.
    \item if $a'$ is an lower bound of $A$, then $a' \le a$.
  \end{enumerate}
\end{definition*}

\section{Limit}

\begin{definition*}
  $\lim\limits_{x\to a} f(x) = l$ means: for any $\eps > 0$, there exists $\delta > 0$ such that $0 < |x-a| < \delta$, then $|f(x)-l| < \eps$.
\end{definition*}

\begin{definition*}
  $\lim\limits_{x \to a^+} f(x) = l$ means: for any $\eps > 0$, there exists $\delta > 0$ such that $0 < x-a < \delta$, then $|f(x)-l| < \eps$.
\end{definition*}

\begin{definition*}
  $\lim\limits_{x \to a^-} f(x) = l$ means: for any $\eps > 0$, there exists $\delta > 0$ such that $0 < a-x < \delta$, then $|f(x)-l| < \eps$.
\end{definition*}

\begin{definition*}
  $\lim\limits_{x \to \infty} f(x) = l$ means: for any $\eps > 0$, there exists $R > 0$ such that $x > R$, then $|f(x)-l| < \eps$.
\end{definition*}

\begin{definition*}
  Let $\{a_n\}_{n \in \NN}$ be a sequence of real numbers. We say that $\lim\limits_{n \to \infty} a_n = l$ if for any $\eps > 0$, there exists $N \in \NN$ such that $n > N$, then $|a_n-l| < \eps$.
\end{definition*}

\section{Continuous Function}

\begin{definition*}
  Let $f: \RR \to \RR$.
  $f$ is continuous at $a$ if $\lim\limits_{x \to a} f(x) = f(a)$.
\end{definition*}

\begin{definition*}
  Let $f: \RR \to \RR$, and $a < $ be real numbers.
  \begin{enumerate}
    \item We say $f$ is continuous on $(a, b)$ if $f$ is continuous at $x$ for every $x \in (a, b)$
    \item We say $f$ is continuous on $[a, b]$ if $f$ is continuous on $(a, b)$ and $\lim\limits_{x \to a^+} f(x) = f(a)$ and $\lim\limits_{x \to b^-} f(x) = f(b)$.
  \end{enumerate}
\end{definition*}

\section{3 Hard theorems}

\begin{theorem*}[Intermediate Value Theorem]
  Let $f : \RR \to \RR$ be continuous on $[a, b]$ for $a < b$. Suppose $f(a) < 0 < f(b)$ then there exists $\xi \in (a, b)$ such that $f(\xi) = 0$.
\end{theorem*}

\begin{theorem*}
  Let $f : \RR \to \RR$ be continuous on $[a, b]$ for $a < b$. Then $f$ is bounded above on $[a, b]$.
  i.e, there exists $M \in \RR$ such that $f(x) \le M$ for all $x \in [a, b]$.
\end{theorem*}

\begin{theorem*}
  Let $f : \RR \to \RR$ be continuous on $[a, b]$ for $a < b$. Then there exists $\xi \in [a, b]$ such that $f(x) \le f(\xi)$ for all $x \in [a, b]$.
  i.e., $f(\xi) = \sup \{f(x) : x \in [a, b]\}$
\end{theorem*}

\section{Uniform Continuity}

\begin{definition*}
  Let $f : \RR \to \RR$.
  We say that $f$ is uniformly continuous on an interval $A$ if for all $\eps > 0$, there exists $\delta = \delta(\eps) > 0$ such that
  $|x-y| < \delta$ and $x, y \in A$ implies $|f(x)-f(y)| < \eps$.
\end{definition*}

\begin{theorem*}
  If $f$ is continuous on $[a, b]$, then $f$ is uniformly continuous on $[a, b]$.
\end{theorem*}

\section{Differentiation}

\begin{definition*}
  $f$ is differentiable at the point $a$ if $\lim\limits_{x \to a} \frac{f(x)-f(a)}{x-a}$ exists. We call this limit $f'(a)$.
\end{definition*}

\begin{theorem*}
  Let $f$ be differentiable at $a$ and $f$ has a maximum at $a$. Then $f'(a) = 0$. 
\end{theorem*}

\begin{theorem*}[Rolle's Theorem]
  If $f$ is continuous on $[a,b]$ and differentiable on $(a, b)$, and $f(a) = f(b)$, then
  there is a number $x$ in $(a, b)$ such that $f'(x) = 0$.
\end{theorem*}

\begin{theorem*}[Mean Value Theorem]
  If $f$ is continuous on $[a,b]$ and differentiable on $(a, b)$, then there is a number $x$ in $(a, b)$ such that
    $f(b) - f(a) = f'(x)(b-a)$
\end{theorem*}

\section{Inverse function}

\begin{theorem*}
  If $f$ is increasing on some interval the it has an inverse function $f^{-1}$.
\end{theorem*}

\begin{theorem*}
  If $f$ be (strictly) increasing on $[a, b]$ and $f'(x_0)$ exists for $x_0 \in (a, b)$ and $f'(x_0) \neq 0$ then 
  $f^{-1}$ is differentiable at $f(x_0)$ and $f^{-1}(f(x_0)) = \frac{1}{f'(x_0)}$. 
\end{theorem*}

\section{Integration}
\begin{definition*}[Partition]
  A partition of $[a, b]$ (called $P$) is a finite collection of distinct numbers in $[a, b]$. 
  which contains the end point $a$ and $b$;
\end{definition*}

\begin{definition*}[Lower Riemann Sum]
  Let $P = \{a = x_0 < x_1 < \dotsc < x_N = b \}$ be a partition of $[a, b]$.
  The lower Riemann sum of $f$ on $P$ is defined as
  \[ L(f, P) = \sum_{j=1}^N (x_j - x_{j-1}) \cdot m_j \]
  where $\displaystyle m_j = \inf_{t \in {[x_{j-1}, x_j]}} f(t)$.
\end{definition*}

\begin{definition*}[Upper Riemann Sum]
  Let $P = \{a = x_0 < x_1 < \dotsc < x_N = b \}$ be a partition of $[a, b]$.
  The upper Riemann sum of $f$ on $P$ is defined as
  \[ U(f, P) = \sum_{j=1}^N (x_j - x_{j-1}) \cdot M_j \]
  where $\displaystyle M_j = \sup_{t \in {[x_{j-1}, x_j]}} f(t)$.
\end{definition*}

\begin{theorem*}
  If $\tilde{P} \supseteq P$ is a partition of $[a, b]$ then \[L(f, P) \le L(f, \tilde{P})\] \[U(f, P) \ge U(f, \tilde{P})\]
\end{theorem*}

\begin{theorem*}
  Let $P_1$ and $P_2$ be partitions of $[a, b]$.
  Let $f$ be a function which is bounded on $[a, b]$ then 
  \[L(f, P_1) \le U(f, P_2)\]
\end{theorem*}

\begin{definition*}
  A function $f$ which is bounded on $[a, b]$ is \textbf{integrable} on $[a, b]$ if 
  \[\sup_P L(f, P) = \inf_P U(f, P)\]
  In this case, this common number is called the \textbf{integral} of $f$ on $[a, b]$ and is denoted by \[\int_a^b f\].
\end{definition*}

\begin{theorem*}
  If $f$ is bounded on $[a, b]$ then $f$ is integrable on $[a, b]$ if and only if for every $\eps > 0$ there is a partition $P$ of $[a, b]$ such that
  \[ U(f, P) - L(f, P) < \eps \]
\end{theorem*}

\begin{theorem*}
  If $f$ is continuous on $[a, b]$ then $f$ is integrable on $[a, b]$.
\end{theorem*}

\begin{theorem*}
  Let $a < c < b$. If $f$ is integrable on $[a, b]$, then $f$ is integrable on $[a, c]$ and on $[c, b]$.
  Conversely, if $f$ is integrable on $[a, c]$ and on $[c, b]$, then $f$ is integrable on $[a, b]$.
  Finally, if $f$ is integrable on $[a, b]$, then
  \[\int_a^b f = \int_a^c f + \int_c^b f\]
\end{theorem*}

\begin{theorem*}
  If $f$ and $g$ are integrable on $[a, b]$, then $f + g$ is integrable on $[a, b]$ and 
  \[ \int_a^b (f + g) = \int_a^b f + \int_a^b g \]
\end{theorem*}

\begin{theorem*}
  If $f$ is integrable on $[a, b]$, then for any number $c$, the function $cf$ is integrable on $[a, b]$ and 
  \[ \int_a^b cf = c \cdot\int_a^b f \]
\end{theorem*}

\begin{theorem*}
  If $f$ is integrable on $[a, b]$ and $F$ is defined on $[a, b]$ By
  \[F(x) = \int_a^x f\]
  then $F$ is continuous on $[a, b]$.
\end{theorem*}
\section{Mean Value Theorem for Integrals}
\begin{theorem*}
  Suppose $f$ is integrable on $[a, b]$ and that $m \le f(x) \le M$ for all $x$ in $[a, b]$.
  Then 
  \[m(b-a) \le \int_a^b f \le M(b-a)\]
\end{theorem*}
\begin{theorem*}
  If $f$ is continuous on $[a, b]$ then there is a point $\xi \in [a, b]$ such that 
  \[\int_a^b f = (b-a)f(\xi)\]
\end{theorem*}
\begin{theorem*}
  Given a function $g$ that is integrable and non-negative. Given a continuous function $f$ on $[a, b]$ then
  there is a point $\xi \in [a, b]$ such that 
  \[ \int_a^b f(x)g(x)\,dx = f(\xi)\int_a^b g(x)\,dx\]
\end{theorem*}

\section{The Fundamental Theorem of Calculus}

\begin{theorem*}
  Let $f$ be integrable on $[a, b]$, and define $F$ on $[a, b]$ by
  \[F(x) = \int_a^x f\]
  If $f$ is continuous at $c$ in $[a, b]$, then $F$ is differentiable at $c$ and 
  \[F'(c) = f(c)\]
\end{theorem*}

\begin{theorem*}
  If $f$ is integrable on $[a, b]$ and $f = g'$ for some function $g$, assume $g'$ is continuous function then 
  \[\int_a^b f = g(b) - g(a)\]
\end{theorem*}

\begin{theorem*}
  If $f$ is integrable on $[a, b]$ and $f = g'$ for some function $g$, then 
  \[\int_a^b f = g(b) - g(a)\]
\end{theorem*}

\section{Rule on Integration}

\begin{theorem*}[Substitution Rule]
  If $g$ is defined on $[a, b]$ and differentiable, and $g'$ is continuous, given $f$ which is defined on an interval containing the range of $g$ and $f$ is continuous.
  Then
  \[\int_a^bf(g(t))g'(t)\,dt = \int_{g(a)}^{g(b)}f(u)\,du\]
\end{theorem*}

\begin{theorem*}[Integration by Parts]
  Given $f, g$ which are differentiable, with $f'$ and $g'$ are continuous on $[a, b]$.
  Then 
  \[\int_a^b f(x)g'(x)\,dx = f(b)g(b) - f(a)g(a) - \int_a^b f'(x)g(x)\,dx\]
\end{theorem*}

\section{Taylor's Theorem}
Given $f$ which is differentiable $n+1$-time, i.e, $f^{(n+1)}$ is continuous 
\[f(x) = T_n(x, a) + R_n(x, a)\]
where 
\[T_n(x, a) = \sum_{k=0}^{n} \frac{f^{(k)}(a)}{k!}(x-a)^n\]
and
\begin{align*}
  R_n(x, a) &= \frac{(x-a)^{(n+1)}}{n!}\int_0^1(1-t)^nf^{(n+1)}(a+t(x-a))\, dt \\
  &= \frac{1}{n!}\int_a^u(x-u)^nf^{(n+1)}(u)\, du \\
  &= \frac{(x-a)^{n+1}}{(n+1)!}f^{(n+1)}(\xi) \\
\end{align*}
where $\xi \in [a, u]$